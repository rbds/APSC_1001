%%%%%%%%%%%%%%%%%%%%%%%%%%%%%%%%%%%%%%%%%
% Short Sectioned Assignment
% LaTeX Template
% Version 1.0 (5/5/12)
%
% This template has been downloaded from:
% http://www.LaTeXTemplates.com
%
% Original author:
% Frits Wenneker (http://www.howtotex.com)
%
% License:
% CC BY-NC-SA 3.0 (http://creativecommons.org/licenses/by-nc-sa/3.0/)
%
%%%%%%%%%%%%%%%%%%%%%%%%%%%%%%%%%%%%%%%%%

%----------------------------------------------------------------------------------------
%	PACKAGES AND OTHER DOCUMENT CONFIGURATIONS
%----------------------------------------------------------------------------------------

\documentclass[paper=a4, fontsize=11pt]{scrartcl} % A4 paper and 11pt font size

\usepackage[T1]{fontenc} % Use 8-bit encoding that has 256 glyphs
%\usepackage{fourier} % Use the Adobe Utopia font for the document - comment this line to return to the LaTeX default
\usepackage[english]{babel} % English language/hyphenation
\usepackage{amsmath,amsfonts,amsthm} % Math packages

\usepackage{lipsum} % Used for inserting dummy 'Lorem ipsum' text into the template

\usepackage{sectsty} % Allows customizing section commands
\allsectionsfont{ \normalfont\scshape} % Make all sections centered, the default font and small caps

\usepackage{graphicx}
\usepackage{booktabs}
\usepackage[framed,numbered,autolinebreaks,useliterate]{mcode}

\usepackage{fancyhdr} % Custom headers and footers
\pagestyle{fancyplain} % Makes all pages in the document conform to the custom headers and footers
\fancyhead{} % No page header - if you want one, create it in the same way as the footers below
\fancyfoot[L]{} % Empty left footer
\fancyfoot[C]{} % Empty center footer
\fancyfoot[R]{\thepage} % Page numbering for right footer
\renewcommand{\headrulewidth}{0pt} % Remove header underlines
\renewcommand{\footrulewidth}{0pt} % Remove footer underlines
\setlength{\headheight}{13.6pt} % Customize the height of the header

\numberwithin{equation}{section} % Number equations within sections (i.e. 1.1, 1.2, 2.1, 2.2 instead of 1, 2, 3, 4)
\numberwithin{figure}{section} % Number figures within sections (i.e. 1.1, 1.2, 2.1, 2.2 instead of 1, 2, 3, 4)
\numberwithin{table}{section} % Number tables within sections (i.e. 1.1, 1.2, 2.1, 2.2 instead of 1, 2, 3, 4)

\setlength\parindent{0pt} % Removes all indentation from paragraphs - comment this line for an assignment with lots of text

%----------------------------------------------------------------------------------------
%	TITLE SECTION
%----------------------------------------------------------------------------------------

\newcommand{\horrule}[1]{\rule{\linewidth}{#1}} % Create horizontal rule command with 1 argument of height

\title{	
\normalfont \normalsize 
\textsc{APSC 1001, George Washington University} \\ [25pt] % Your university, school and/or department name(s)
\horrule{0.5pt} \\[0.4cm] % Thin top horizontal rule
\huge APSC 1001 Introduction to Plotting in Matlab \\ % The assignment title
\horrule{2pt} \\[0.5cm] % Thick bottom horizontal rule
}

\author{\normalsize Randy Schur } % Your name

%\normalsize Week 1 Handout
\date{\normalsize \today } % Today's date or a custom date


\begin{document}

\maketitle % Print the title

\subsection{Useful Functions}
There are a few functions that are in MATLAB which are not specific to plotting, but will be both helpful and frequently used.

\begin{itemize}
\item \textbf{help \textit{function}} In the MATLAB command window, you can type \textit{help} then the name of any function. This will bring up documentation on the different uses for that function, the arguments it takes, what it returns, and often example uses and related functions. Further help and documentation is available online, or by typing \textit{doc} plus the name of the function. 
\item \textbf{linspace(m,n,N)} The linspace function creates an array of size $m \times n$ with $N$ linearly spaced entries.
\begin{verbatim}
>>t = linspace(0, 1, 11) %notice that 11 numbers are required to create 10 steps between 0 and 1 
t =
  Columns 1 through 7
         0    0.1000    0.2000    0.3000    0.4000    0.5000    0.6000

  Columns 8 through 11
    0.7000    0.8000    0.9000    1.0000
\end{verbatim} 
 
\item \textbf{colon operator}	The colon, :, is used in Matlab to denote ranges of numbers. It can also be used to create arrays. The command:
\begin{verbatim}
>>t = 0:0.1:1	
\end{verbatim} 
will create an array with entries ranging from 0 to 1 in step sizes of 0.1. 
This will have the same output as the previous \textit{linspace(m, n, N)} example
\end{itemize}

\subsection{Plot Command}
The plot command in MATLAB looks like the following: 
\mcode{plot(x, y, optional_arguments)}
In order to plot arrays in Matlab, they must be of the same length.
The optional arguments cover 




\end{document}