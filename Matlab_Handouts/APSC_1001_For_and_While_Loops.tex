%%%%%%%%%%%%%%%%%%%%%%%%%%%%%%%%%%%%%%%%%
% Short Sectioned Assignment
% LaTeX Template
% Version 1.0 (5/5/12)
%
% This template has been downloaded from:
% http://www.LaTeXTemplates.com
%
% Original author:
% Frits Wenneker (http://www.howtotex.com)
%
% License:
% CC BY-NC-SA 3.0 (http://creativecommons.org/licenses/by-nc-sa/3.0/)
%
%%%%%%%%%%%%%%%%%%%%%%%%%%%%%%%%%%%%%%%%%

%----------------------------------------------------------------------------------------
%	PACKAGES AND OTHER DOCUMENT CONFIGURATIONS
%----------------------------------------------------------------------------------------

\documentclass[paper=a4, fontsize=11pt]{scrartcl} % A4 paper and 11pt font size

\usepackage[T1]{fontenc} % Use 8-bit encoding that has 256 glyphs
%\usepackage{fourier} % Use the Adobe Utopia font for the document - comment this line to return to the LaTeX default
\usepackage[english]{babel} % English language/hyphenation
\usepackage{amsmath,amsfonts,amsthm} % Math packages

\usepackage{lipsum} % Used for inserting dummy 'Lorem ipsum' text into the template

\usepackage{sectsty} % Allows customizing section commands
\allsectionsfont{ \normalfont\scshape} % Make all sections centered, the default font and small caps

\usepackage{graphicx}
\usepackage{booktabs}
\usepackage[framed,numbered,autolinebreaks,useliterate]{mcode}

\usepackage{fancyhdr} % Custom headers and footers
\pagestyle{fancyplain} % Makes all pages in the document conform to the custom headers and footers
\fancyhead{} % No page header - if you want one, create it in the same way as the footers below
\fancyfoot[L]{} % Empty left footer
\fancyfoot[C]{} % Empty center footer
\fancyfoot[R]{\thepage} % Page numbering for right footer
\renewcommand{\headrulewidth}{0pt} % Remove header underlines
\renewcommand{\footrulewidth}{0pt} % Remove footer underlines
\setlength{\headheight}{13.6pt} % Customize the height of the header

\numberwithin{equation}{section} % Number equations within sections (i.e. 1.1, 1.2, 2.1, 2.2 instead of 1, 2, 3, 4)
\numberwithin{figure}{section} % Number figures within sections (i.e. 1.1, 1.2, 2.1, 2.2 instead of 1, 2, 3, 4)
\numberwithin{table}{section} % Number tables within sections (i.e. 1.1, 1.2, 2.1, 2.2 instead of 1, 2, 3, 4)

\setlength\parindent{0pt} % Removes all indentation from paragraphs - comment this line for an assignment with lots of text

%----------------------------------------------------------------------------------------
%	TITLE SECTION
%----------------------------------------------------------------------------------------

\newcommand{\horrule}[1]{\rule{\linewidth}{#1}} % Create horizontal rule command with 1 argument of height

\title{	
\normalfont \normalsize 
\textsc{APSC 1001, George Washington University} \\ [25pt] % Your university, school and/or department name(s)
\horrule{0.5pt} \\[0.4cm] % Thin top horizontal rule
\huge APSC 1001 For Loops and While Loops\\ % The assignment title
\horrule{2pt} \\[0.5cm] % Thick bottom horizontal rule
}

\author{\normalsize Randy Schur } % Your name

%\normalsize Week 1 Handout
\date{\normalsize \today } % Today's date or a custom date


\begin{document}

\maketitle % Print the title
The programs we write are containted in script files, which act as a list of commands for MATLAB to execute. 
It is often useful to execute sections of our code more than once. 
We can do this without repeating lines of code by using a for loop or a while loop.

\section{For Loops}
 A for loop is useful for repeating a section of code a known number of times. The for loop takes a variable called an index, and repeats the code with a different index each time. Most commonly we start the index at 1 and increase it by 1 each time, but this isn't the only way to write a for loop. \\
 
Take a look at the two examples below, based on examples from the MATLAB documentation.
\lstinputlisting[]{for_loop_example.m}
This example calculates a power of 2. 
There are a few important items to notice about this simple block of code.
First, the variable t that is used inside the loop is initialized outside the loop. 
Next, t is overwritten at each iteration of the loop. 
The loop runs 10 times (from j=1 until j=10), so it will give us $2^{10}$. 
We could caclculate $2^n$ for any whole number n by changing the loop conditions from 10 to n.


\lstinputlisting[]{for_loop_example_2.m}
This for loop has several differences from the first example. 
Again, the variable x is initialized, but this time it is an array of the correct size. This isn't necessary, but it is recommended for efficiency.
The loop runs once for k=2, then for k=3, then k=3, etc.
This time, the variable x is not overwritten. Instead each iteration of the loop is saved as a different value in the vector x.
Notice that the current value of x is dependent on a previous value. This is one main reason why for loops are useful; in this case the entries in x need to be calculated in order and can't all be found at once.
Try typing these examples into MATLAB yourself and adjusting the parameters of the for loop.

\section{While Loops}
While loops are used when the number of iterations needed is unknown. 
While loops take a condition and run as long as that condition is true.
In this case, the variable used in the index needs to be initialized outside the loop.
Importantly, they do not index on their own, so you need to be sure to adjust the condition inside the loop.\\
 
For example, what are the first 10 prime numbers? We can find out with a while loop. 
\lstinputlisting[]{while_loop_example.m}


Note that it is possible to create a while loop tha never terminates on its own. 
This can happen if you do not update the variable used as the loop condition.
You must be careful not to do this when writing code. 
If you do create such a situation accidentally, in MATLAB you can press Ctrl+c to terminate the current operation.\\

For loops and while loops are powerful tools than can be found in almost any programming language. 
Learning to use them effectively will help to use more of MATLAB's potential; the reason we use MATLAB instead of just a graphing calculator is the availability of programmatic tools like these!
\newpage
\section*{References}
MATLAB Documentation, 2015.

\begin{verbatim} http://www.mathworks.com/help/matlab/matlab_prog/loop-control-statements.html
\end{verbatim}
\end{document}
