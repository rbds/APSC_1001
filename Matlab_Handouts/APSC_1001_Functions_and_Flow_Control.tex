%%%%%%%%%%%%%%%%%%%%%%%%%%%%%%%%%%%%%%%%%
% Short Sectioned Assignment
% LaTeX Template
% Version 1.0 (5/5/12)
%
% This template has been downloaded from:
% http://www.LaTeXTemplates.com
%
% Original author:
% Frits Wenneker (http://www.howtotex.com)
%
% License:
% CC BY-NC-SA 3.0 (http://creativecommons.org/licenses/by-nc-sa/3.0/)
%
%%%%%%%%%%%%%%%%%%%%%%%%%%%%%%%%%%%%%%%%%

%----------------------------------------------------------------------------------------
%	PACKAGES AND OTHER DOCUMENT CONFIGURATIONS
%----------------------------------------------------------------------------------------

\documentclass[paper=a4, fontsize=11pt]{scrartcl} % A4 paper and 11pt font size

\usepackage[T1]{fontenc} % Use 8-bit encoding that has 256 glyphs
%\usepackage{fourier} % Use the Adobe Utopia font for the document - comment this line to return to the LaTeX default
\usepackage[english]{babel} % English language/hyphenation
\usepackage{amsmath,amsfonts,amsthm} % Math packages

\usepackage{lipsum} % Used for inserting dummy 'Lorem ipsum' text into the template

\usepackage{sectsty} % Allows customizing section commands
\allsectionsfont{ \normalfont\scshape} % Make all sections centered, the default font and small caps

\usepackage{graphicx}
\usepackage{booktabs}
\usepackage[framed,numbered,autolinebreaks,useliterate]{mcode}

\usepackage{fancyhdr} % Custom headers and footers
\pagestyle{fancyplain} % Makes all pages in the document conform to the custom headers and footers
\fancyhead{} % No page header - if you want one, create it in the same way as the footers below
\fancyfoot[L]{} % Empty left footer
\fancyfoot[C]{} % Empty center footer
\fancyfoot[R]{\thepage} % Page numbering for right footer
\renewcommand{\headrulewidth}{0pt} % Remove header underlines
\renewcommand{\footrulewidth}{0pt} % Remove footer underlines
\setlength{\headheight}{13.6pt} % Customize the height of the header

\numberwithin{equation}{section} % Number equations within sections (i.e. 1.1, 1.2, 2.1, 2.2 instead of 1, 2, 3, 4)
\numberwithin{figure}{section} % Number figures within sections (i.e. 1.1, 1.2, 2.1, 2.2 instead of 1, 2, 3, 4)
\numberwithin{table}{section} % Number tables within sections (i.e. 1.1, 1.2, 2.1, 2.2 instead of 1, 2, 3, 4)

\setlength\parindent{0pt} % Removes all indentation from paragraphs - comment this line for an assignment with lots of text

%----------------------------------------------------------------------------------------
%	TITLE SECTION
%----------------------------------------------------------------------------------------

\newcommand{\horrule}[1]{\rule{\linewidth}{#1}} % Create horizontal rule command with 1 argument of height

\title{	
\normalfont \normalsize 
\textsc{APSC 1001, George Washington University} \\ [25pt] % Your university, school and/or department name(s)
\horrule{0.5pt} \\[0.4cm] % Thin top horizontal rule
\huge APSC 1001 Functions and If Statements\\ % The assignment title
\horrule{2pt} \\[0.5cm] % Thick bottom horizontal rule
}

\author{\normalsize Randy Schur } % Your name

%\normalsize Week 1 Handout
\date{\normalsize \today } % Today's date or a custom date


\begin{document}

\maketitle % Print the title

\section{Functions}
Up to this point, all of our .m MATLAB files have been scripts, which execute a set of commands in order.  There is another type of .m file called a function, which allows us to create our own MATLAB function. Much like the built in MATLAB functions that we use on a regular basis, the functions we write can take input and output arguments of scalars, vectors, or matrices. A function file contains at a minimum the following code:

\lstinputlisting[]{function_example.m}

There are several important parts here. 
First, the file must be declared a function right at the beginning; this is how MATLAB can tell the difference between a function or a script .m file. 
Next come the output arguments (if any), then an '=' sign, then the name of the function, then the input arguments (if any). 
This format mirrors the way you would call the function in the command window. 

\subsection{Function example}
Below is an example function file that takes in the side lengths of a triangle and returns the internal angles. 
\lstinputlisting[]{triangle.m}

Notice the comments in the function at the very beginning. 
These are what come up when you type 'help' into the command window, so make them useful! 
Additional comments throughout the function can also be helpful to anyone, including yourself, who looks at the code inside the function.\\

Try writing your own function to do the reverse - it should take as input arguments the internal angles and return the side lengths of a triangle.


\section{Flow Control}
There is one line in the code above that might not be familiar, and that is the \textit{if statement}. 
Often when programming, we want to do something more involved than simply executing a list of commands. 
Sometimes, we might want to execute a block of code only if a certain condition is met.
Other times, we want to execute the same block of code multiple times.
This is generally called flow control of a program, and the most common methods are the \textit{if-else statement}, the \textit{for loop}, and the \textit{while loop}.

\subsection{If Statement}
The \textit{if statement} has two components: the condition and the execution.
The statement says \textit{if} the condition is true, \textit{then} execute the code inside. Take a look at the code below for an example. 
\lstinputlisting[]{if_example.m}

There are two other components to an if statement. 
The \textit{else if} is equivalent to combining mulitple if statements. 
The \textit{else} statement is a catch-all category for any situation in which none of the previous conditions are met.
For example, see the following code which takes an input grade and outputs the corresponding letter grade.
\lstinputlisting[]{grade_calc.m}
What happens when we give the function an input of 95? What about 77? What about 62?

\end{document}
