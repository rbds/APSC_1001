%=======================02-713 LaTeX template, following the 15-210 template==================
%
% You don't need to use LaTeX or this template, but you must turn your homework in as
% a typeset PDF somehow.
%
% How to use:
%    1. Update your information in section "A" below
%    2. Write your answers in section "B" below. Precede answers for all 
%       parts of a question with the command "\question{n}{desc}" where n is
%       the question number and "desc" is a short, one-line description of 
%       the problem. There is no need to restate the problem.
%    3. If a question has multiple parts, precede the answer to part x with the
%       command "\part{x}".
%    4. If a problem asks you to design an algorithm, use the commands
%       \algorithm, \correctness, \runtime to precede your discussion of the 
%       description of the algorithm, its correctness, and its running time, respectively.
%    5. You can include graphics by using the command \includegraphics{FILENAME}
%
\documentclass[11pt]{article}
\usepackage{amsmath,amssymb,amsthm}
\usepackage{graphicx}
%\usepackage[margin=1in]{geometry}
\usepackage{fancyhdr}
\usepackage[framed,numbered,autolinebreaks,useliterate]{mcode}
\setlength{\parindent}{0pt}
\setlength{\parskip}{5pt plus 1pt}
\setlength{\headheight}{16pt}
\newcommand\question[2]{\vspace{.25in}\textbf{#1: #2}\vspace{.5em}\hrule\vspace{.10in}}
\renewcommand\part[1]{\vspace{.10in}\textbf{(#1)}}
\pagestyle{fancyplain}
\lhead{\Large \textbf{APSC 1001}}
\chead{\LARGE \textbf{HW3}}
\rhead{Due: 10/16/15}
\begin{document}\raggedright

\question{1}{Functions} 
Your functions should have the help comments filled out as well as comments throughout the file.

\part{a} Create a function that takes as input distance in meters and outputs distance in inches.

\part{b} Create a function that takes as input mass in kilograms and outputs weight in pounds.

\part{c} Create a script that tests each of your functions. 
The input to the function in part a should be the vector [72 63 57 46 12 1]. 
The input to the funtion in part b should be [12 16 18 32 46 1].

The functions cannot be defined within your script, but should be saved in the same location as your script. You will turn in a single .pdf with all questions answered, plus separate .m function files. 

\question{2}{If Statements}
\part{a} Create a 1x100 vector of random numbers with a Gaussian distribution and a mean of 1.
Find the actual mean of the entries in this vector. 
Using an if statement, output to the command line what the mean is and whether it is less than, greater than, or equal to 1.

\vspace{1.0em}
\textbf{Directions}
Both questions should be answered in a single script using cell mode. Please see the example MATLAB files that have been posted to Blackboard from past weeks for examples in formatting. Any functions will be turned in as separate .m files.

You must turn this assignment in to Blackboard as a published pdf. Create a script to complete the homework assignment, taking care to control what is output to the command window. The code should be \textbf{well commented} so that it is easy to follow along. 


\end{document}
