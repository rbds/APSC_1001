%=======================02-713 LaTeX template, following the 15-210 template==================
%
% You don't need to use LaTeX or this template, but you must turn your homework in as
% a typeset PDF somehow.
%
% How to use:
%    1. Update your information in section "A" below
%    2. Write your answers in section "B" below. Precede answers for all 
%       parts of a question with the command "\question{n}{desc}" where n is
%       the question number and "desc" is a short, one-line description of 
%       the problem. There is no need to restate the problem.
%    3. If a question has multiple parts, precede the answer to part x with the
%       command "\part{x}".
%    4. If a problem asks you to design an algorithm, use the commands
%       \algorithm, \correctness, \runtime to precede your discussion of the 
%       description of the algorithm, its correctness, and its running time, respectively.
%    5. You can include graphics by using the command \includegraphics{FILENAME}
%
\documentclass[11pt]{article}
\usepackage{amsmath,amssymb,amsthm}
\usepackage{graphicx}
%\usepackage[margin=1in]{geometry}
\usepackage{fancyhdr}
\usepackage[framed,numbered,autolinebreaks,useliterate]{mcode}
\setlength{\parindent}{0pt}
\setlength{\parskip}{5pt plus 1pt}
\setlength{\headheight}{16pt}
\newcommand\question[2]{\vspace{.25in}\textbf{#1: #2}\vspace{.5em}\hrule\vspace{.10in}}
\renewcommand\part[1]{\vspace{.10in}\textbf{(#1)}}
\pagestyle{fancyplain}
\lhead{\Large \textbf{APSC 1001}}
\chead{\LARGE \textbf{HW2}}
\rhead{Due: 10/9/15}
\begin{document}\raggedright
%Section A==============Change the values below to match your information==================
\newcommand\NAME{Carl Kingsford}  % your name
\newcommand\ANDREWID{ckingsf}     % your andrew id
\newcommand\HWNUM{1}              % the homework number
%Section B==============Put your answers to the questions below here=======================

% no need to restate the problem --- the graders know which problem is which,
% but replacing "The First Problem" with a short phrase will help you remember
% which problem this is when you read over your homeworks to study.

\question{1}{Curve Fitting} 
\part{a} Import the data from the file "Data\_HW2.csv", which can be found on blackboard. You can use the following code:
\lstinputlisting[firstline=5, lastline=15]{HW2_example_code.m}
Here, you are importing the data, then saving each column as a separate array. X1 and Y1 contain one set of data, and X2 and Y2 contain a separate set of data.

\part{b} For each data set, create a script to find a curve that fits the data. Plot the data as discrete points, and your curve as a continuous function. Report on your goodness of fit, and give one scenario where the data may have come from.

\vspace{2.0em}
\textbf{Directions}
You must turn this assignment in to Blackboard as a published pdf. Create a script to complete the homework assignment, taking care to control what is output to the command window. The code should be \textbf{well commented} so that it is easy to follow along. See directions for publishing to pdf below. 

\begin{enumerate} \itemsep -2pt
\item Go to the publish tab in MATLAB
\item Select the drow down under 'Publish'
\item Edit Publishing Options
\item Output file format should be '.pdf'. This is the only change you should need to make.
\item Press publish, and save the resulting pdf with your name and the assignment number in the title.
\end{enumerate}


\end{document}
